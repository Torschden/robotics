%% cleanthesis-doc.tex
%% Copyright 2015 R. Langner
%
% This work may be distributed and/or modified under the
% conditions of the LaTeX Project Public License, either version 1.3
% of this license or (at your option) any later version.
% The latest version of this license is in
%   http://www.latex-project.org/lppl.txt
% and version 1.3 or later is part of all distributions of LaTeX
% version 2005/12/01 or later.
%
% This work has the LPPL maintenance status `maintained'.
%
% The Current Maintainer of this work is R. Langner.
%
% This work consists of all files listed in MANIFEST.md.
%
\documentclass{ltxdockit}
\usepackage{btxdockit}
\usepackage[utf8]{inputenc}
\usepackage[american]{babel}
\usepackage[strict]{csquotes}
\usepackage{tabularx}
\usepackage{longtable}
\usepackage{amsmath}
\usepackage{booktabs}
\usepackage{shortvrb}
\usepackage{pifont}
\usepackage[pdftex]{graphicx}
\usepackage[backend=biber]{biblatex}
\usepackage{listings}
\usepackage{hyperref}

\addbibresource{anti_sloshing.bib}
\graphicspath{/images/}

\DeclareSymbolFont{mycmsy}{OMS}{cmsy}{m}{n}

\DeclareMathSymbol{\cmsyD}{\mathalpha}{mycmsy}{'104}
\DeclareMathSymbol{\cmsyT}{\mathalpha}{mycmsy}{'124}
\DeclareMathSymbol{\cmsyV}{\mathalpha}{mycmsy}{'126}

\rcsid{$Id: cleanthesis.tex,v 0.0.1 2025/07/23 11:00:00 derric stable $}

\newcommand*{\cleanthesis}{\emph{Clean Thesis}\xspace}
\newcommand*{\cthesishome}{http://cleanthesis.der-ric.de/}
%\newcommand*{\cthesisctan}{http://www.ctan.org/tex-archive/macros/latex/contrib/../}

\titlepage{%
  title={DPE Framework Meeting Notes}
  url={\cthesishome},
  author={DPE},
  email={kai.bauer2@kit.edu},
  revision={\rcsrevision},
  date={\rcstoday}}

\hypersetup{%
  pdftitle={DPE Framework Meeting Notes}
  pdfauthor={DPE},
  pdfkeywords={tex, latex, thesis, style}}

%\setcounter{secnumdepth}{4}

\begin{document}

\tableofcontents
%\listoftables

\section{Framework}
\label{sec:framework}

logs of the meetings regarding the DPE Framework (previouslly \textit{maritime framework})

\subsection{Meeting}
\label{sec:framework:meeting}
\subsubsection{Meeting 1, 16.07.2025}
\begin{itemize}
  \item general usage of the framework
  \item regular wednesday meetings
\end{itemize}

\subsubsection{Meeting 2, 23.07.2025}
\begin{itemize}
  \item Framwork in top level directory in Gitlab
  \item control framewrok is one repository instead of 10 different ones
  \item trunk based development (short lived branches, issue based)
  \item name commits after commit guidelines for automatic changelog generation
  \item docstrings for documentation, stick to matlab style
  \item typing: explanation, 
\end{itemize}

\subsubsection{Meeting 3, 30.07.2025}
\begin{itemize}
  \item CI pipeline mentioned
  \item Types implementation: abstraction via plugins, keeping he CF lean, get more specific in the plugins
  \item Tests eg instanciate types and execute examples (testCase.xyz) 
  \item Control loop: Blocks are optional, but the naming of the signals is strict, global bus for enviroment data (disturbance)
  \item possible combinations of blocks in the control loop
\end{itemize}

difference static reference and trajectory (with constant path)

\subsubsection{Meeting 3, 06.08.2025}
\begin{itemize}
  \item barebone structure in Gitlab
  \item abstract keywords need to be defined to be able to instanciate
  \item simulation
  \item lukas: all possible combinations (way too much), relevant ~20. 
\end{itemize}

\end{document}

